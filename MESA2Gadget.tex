\PassOptionsToPackage{svgnames}{xcolor}
\documentclass{beamer}

%%%%%%%%%%%%%%%%%%%%%%%%
\mode<presentation> 
\usetheme{default}
\beamertemplatenavigationsymbolsempty 



% ----------------------------- COLOR library ----------------------------------
% RGB colors
\definecolor{palegreen}{RGB}{180,250,180}%{152,250,152}%{190,250,208}
\definecolor{normalblack}{RGB}{0,0,0} 
\definecolor{palepink}{RGB}{ 245,198,232} %255, 229, 237} %
\definecolor{maturepink}{RGB}{250,225,250} %pretty accurate
\definecolor{hotpink}{RGB}{249,242,215}
\definecolor{navy}{RGB}{0,0,51}%{26,18,69} % more like very deep violet
%\definecolor{lightblue}{RGB}{126,118,169} % more like light purple marker
\definecolor{lightblue}{RGB}{173,216,230}

\definecolor{forestgreen}{RGB}{0,45,0}%{10,50,10} 

\definecolor{normalwhite}{RGB}{250,250,250} %% "white" works fine

%colors that matter for this scheme
\definecolor{burgundy}{RGB}{120, 0, 43}
\definecolor{lightsalmon}{RGB}{255, 179, 160}
\definecolor{desertblue}{RGB}{60,0,80}



\setbeamercolor{titlelike}{bg=forestgreen, fg=normalwhite }
\setbeamercolor{normal text}{fg=normalwhite,bg=forestgreen}  %%% bg means background slide color
\setbeamercolor{alerted text}{bg=burgundy, fg=normalwhite}


%%%%%%%%%%%%%%%%% packages
\usepackage{graphicx} % Allows including images
\usepackage{booktabs} % Allows the use of \toprule, \midrule and \bottomrule in tables
\usepackage{graphicx,subfigure,geometry}
\usepackage{graphicx}
\DeclareGraphicsExtensions{.pdf,.png,.jpg}
\usepackage{bm}
\usepackage{hyperref}
\hypersetup{colorlinks, linkcolor=normalwhite}
\usepackage{subfigure}
\usepackage{amsmath}
\usepackage{amssymb}
\usepackage{wasysym}
\def\k{0.45}%{0.4} % tweak this for less horizontal space between images / to resize them
\def\vgap{0.2em}%{0em} % tweak this for vertical spac

\usepackage{url}
\usepackage{amsmath}
\usepackage{times}
\usepackage{amsmath,amsthm, amssymb, latexsym}
\usepackage{exscale}
\usepackage{booktabs, array}
\usepackage{graphicx}
\usepackage{bm}
\usepackage{hyperref}
\usepackage{subfigure}
\usepackage{amsmath}
\usepackage{amssymb}
\usepackage{wasysym}




\begin{document}


% %----------------------------------------------------------------------------------------
% %	PRESENTATION SLIDES
% %----------------------------------------------------------------------------------------
%----------------------------------------------------------------------------------------
%	TITLE PAGE
%----------------------------------------------------------------------------------------


\begin{frame}
%\begin{centering}
\vskip4ex
{\Huge MESA2GADGET}
\vskip0.5ex
{\Large Bridging the Gap between 1D and 3D Stellar Models}
%A Software Toolkit for Linking 1D and 3D Stellar Models}
%\end{centering}
\vskip10ex
\flushright
Meridith Joyce %$^{1}$
\vskip0.5ex
\footnotesize Ph.D. Candidate
\vskip0.5ex
\footnotesize Dartmouth
\vskip0.5ex \footnotesize SAAO \& UCT
%\vskip0.5ex \footnotesize UCT
%\vskip3ex

%\large University of the Western Cape
\vskip1ex
\flushright Oct 30$^\text{th}, 2017$
 
 % \vskip1ex
 % \flushleft \footnotesize $^{1}$now accepting postdoc offers
\end{frame}

%-----------------------------------------------------------------------------------
%--------------------------------------------------------------------------------------
\begin{frame}
%\begin{figure}
\frametitle{Why?} 
%{Motivation}
\centering
\includegraphics[width=\columnwidth]{{questionmark}.png} \vskip\vgap
% \begin{itemize}
% 	\item[] AGB stars are interesting because...
% 	\begin{itemize}
% 		\item[] NOT FILLED IN YET
% 		\item[] impact of chemistry on thermal pulsations

% 		\item[] dust grain formation $\rightarrow$ galaxy evolution insights
% 	\end{itemize}
% \end{itemize}
%\end{figure}
\end{frame}

%---------------------------------------------------------------------------------------
%--------------------------------------------------------------------------------------
\begin{frame}

\frametitle{Meet the Codes} 
%: Modules for Experiments in Stellar Astrophysics}
\centering
\begin{figure}
\centering
\includegraphics[width=0.44\columnwidth]{{MESA}.png} \qquad \hskip2ex
\includegraphics[width=0.45\columnwidth]{{Gadget}.png} 
\vskip0.5ex
{\footnotesize
\begin{itemize}
\item[] MESA: Modules for Experiments in Stellar Astrophysics
	\begin{itemize}
		\item[-] \footnotesize 1 dimensional stellar structure solver 
		\item[-] \footnotesize highly customizable physical conditions

	\end{itemize}
\item[] GADGET: GAlaxies with Dark matter and Gas intEracT
	\begin{itemize}
		\item[-] \footnotesize N-body/SPH
		\item[-] \footnotesize typically used for galactic and cosmological simulations
		\item[-] \footnotesize SPH also useful for stellar atmospheres
	\end{itemize}
\end{itemize}	
}
%\includegraphics[width=\columnwidth]{{MESA}.png} 
\end{figure}
\end{frame}


%--------------------------------------------------------------------------------------
% \begin{frame}

% \frametitle{Gadget: Smoothed Particle Hydrodynamics}
% \centering
% \begin{figure}
% \includegraphics[width=\columnwidth]{{Gadget}.png} 
% \end{figure}
% \end{frame}



%--------------------------------------------------------------------------------------
\begin{frame}
\begin{figure}
\frametitle{TP-AGB Models}
\centering
\includegraphics[width=0.88\columnwidth]{{textbook02}.png} 
%\qquad \hskip2ex
%\includegraphics[width=0.44\columnwidth]{{textbook03}.png} 
\vskip0.5ex
% {\small
% Either do this well or don't do it-- see how long I'm being given
% }
\end{figure}
\end{frame}

%---------------------------------------------------------------------------------------
\begin{frame}
\frametitle{\small Method}
{\footnotesize
\begin{itemize}
\item[o] generate MESA density profile


\item[o] subdivide into $k$ regions of varying size $(r_u - r_l)_{k}$ such that the number of particles $N$ is preserved per region and the mass per particle $m_p$ is preserved $\forall k$ 
	\begin{itemize}
		\item[-]\footnotesize requires either numerical integration or model fitting to find the mass contained per region, which determines $N$ (or $m_p$)
	\end{itemize}	

\item[o] distribute the $N$ particles contained in each region $k$ across the surface of a sphere of radius  $r_{\text{mid},k} = \frac{(r_u + r_l)_k}{2}$ 
	\begin{itemize}
		\item[-]\footnotesize care is required in selecting a particle distribution method that will minimize computational artifacts $\rightarrow$ HEALPix
		%\item[-]\footnotesize we use HEALPIX
	\end{itemize}	

\item[o] stack these shells: $\forall k, x = x+x_{k}$ (same for $y, z$) at $r_{\text{mid},k}$
	\begin{itemize}
		\item[-] \footnotesize IMPORTANT! must arbitrarily rotate each shell or the particles will be ordered
	\end{itemize}

\item[o] send final $x, y, z$ arrays to a Gadget initial conditions (IC) generator 

\item[o] VALIDATION!
	\begin{itemize}
		\item[-]\footnotesize Load the ICs! Check: does $\rho$ vs $r_k=\sqrt{x_k^2 + y_k^2 + z_k^2} \forall$ regions $k$ recover the initial radial sampling applied to MESA?
	\end{itemize}

\end{itemize}
}
\end{frame}


%--------------------------------------------------------------------------------------
\begin{frame}
\frametitle{\small Ohlmann et al., 2017}
\begin{figure}
\centering
%\includegraphics[width=0.98\columnwidth]{{Pakmor_density_profiles}.png} \vskip\vgap
\includegraphics[width=0.98\columnwidth]{{Ohlmann_density_profiles}.png} \vskip\vgap
\end{figure}
\end{frame}

%--------------------------------introduce rl, ru here ------------------------------
\begin{frame}
\begin{figure}
\centering
\includegraphics[width=0.98\columnwidth]{{overlay_ru}.png} \vskip\vgap
%\includegraphics[width=0.98\columnwidth]{{stacking_shells}.png} \vskip\vgap
\end{figure}
\end{frame}


%--------------------------------------------------------------------------------------
\begin{frame}
\begin{figure}
%\frametitle{MESA profile processing}
\centering
\includegraphics[width=0.98\columnwidth]{{fit_region_formal}.png} \vskip\vgap
\end{figure}
\end{frame}


%--------------------------------------------------------------------------------------
\begin{frame}
\frametitle{Workflow}
\begin{figure}
\includegraphics[width=0.98\columnwidth]{{flowofcontrol}.png} \vskip\vgap
\end{figure}

\end{frame}

%-------------------------from healpix paper-------------------------------------------------------------
\begin{frame}
\frametitle{\small HEALPix: Hierarchical Equal Area iso-Latitude Pixelization}
\begin{figure}
\centering
\includegraphics[width=0.98\columnwidth]{{tessellation}.png} \vskip\vgap
\end{figure}
\end{frame}

%--------------------------------------------------------------------------------------
\begin{frame}
\begin{figure}
% \frametitle{HEALPix: Hierarchical Equal Area iso-Latitude Pixelization}
\centering
\includegraphics[width=0.98\columnwidth]{{mollview_nest_w_coords}.png} \vskip\vgap
\end{figure}
\end{frame}

%--------------------------------------------------------------------------------------
\begin{frame}
\begin{figure}
\frametitle{HEALPix distribution for arbitrary shell $k$}
\centering
\includegraphics[width=0.98\columnwidth]{{raw_healpix_shell}.png} \vskip\vgap
\end{figure}
\end{frame}


%--------------------------------------------------------------------------------------
\begin{frame}
\frametitle{\small Convergence Criteria: $n_1(r_u)$ and $n_2(r_u)$}
{\small 
Method: Impose two independent constraints and force their equality
%: one physical, deriving from $M, \rho$, and one deriving from modeling limitations (HEALPix)
\vskip0.5ex
HEALPix tessellates a sphere into $12 n^2$ quadrilaterals for $n \in \{ 2^x \}; n,x \in \mathcal{Z}$
\vskip0.5ex
Let $n_1 $ s.t. $n_p = 12 n_1^2$, where $n_p = \frac{M_{\text{shell}}}{m_p}$, with $n_p$ the number of particles per shell
\vskip0.5ex
\small Then $ n_1  = \sqrt{\frac{ M_{\text{shell}}}{12m_p} } $, where $M_{\text{shell}}=M_{\text{shell}}(r_u, r_l)$
\vskip0.5ex
\small Now, let each quadrilateral have width $r_u -r_l$. The surface area of that quadrilateral is $(r_u - r_l)^2$
\vskip0.5ex
\small Simultaneously, the total surface area of the shell $k$ is $4\pi r_{\text{mid}}^2 =4\pi \frac{(r_u + r_l)}{2}^2$ 
\vskip0.5ex
\small HEALPix requires $12n_2^2$ partciles and $12n_2^2$ quadrilaterals via 1 particle per region constraint. Hence, $(r_u - r_l )^2 12 n_2^2 = 4 \pi \frac{(r_u+r_l)}{2}^2$  
\vskip0.5ex
\small  which gives $\rightarrow n_2 = \sqrt{\frac{\pi}{12}} \frac{r_u + r_l}{r_u - r_l}$. 
\vskip0.5ex
\small Because $n_1$ is a monotonically increasing function of $r_u$ and $n_2$ is monotonically decreasing, $\exists r_u$ s.t. $n_1 = n_2$*
\vskip0.5ex
\footnotesize *IGNORING all other constraints on $M_{\text{shell}}$
}
\end{frame}

%--------------------------------------------------------------------------------------
\begin{frame}
\begin{figure}
\frametitle{\small For fixed $r_l$ and increasing $r_u$}
\centering
\includegraphics[width=0.98\columnwidth]{{n1_vs_n2}.png} \vskip\vgap
\end{figure}
\end{frame}



%--------------------------------------------------------------------------------------
\begin{frame}
\begin{figure}
\frametitle{Convergence}
\centering
\includegraphics[width=0.98\columnwidth]{{find_intersection}.png} \vskip\vgap
\end{figure}
\end{frame}





% %--------------------------------------------------------------------------------------
% \begin{frame}
% \begin{figure}
% \frametitle{Effect of ordered shell stacking}
% \centering
% \includegraphics[width=0.98\columnwidth]{{discrete_radial_effects_gadget}.png} \vskip\vgap
% \end{figure}
% \end{frame}


%--------------------------------------------------------------------------------------
\begin{frame}
\begin{figure}
%\frametitle{Final Form (N=8)}
\centering
\includegraphics[width=0.98\columnwidth]{{ordered_AGB_star}.png} \vskip\vgap
\end{figure}
\end{frame}


%--------------------------------------------------------------------------------------
\begin{frame}
\begin{figure}
%\frametitle{MESA profile processing}
\centering
\includegraphics[width=0.98\columnwidth]{{projected_density_AGB140}.png} \vskip\vgap
\end{figure}
\end{frame}

%--------------------------------------------------------------------------------------
\begin{frame}
\begin{figure}
%\frametitle{MESA profile processing}
\centering
\includegraphics[width=0.98\columnwidth]{{smoothed_projected_density_AGB140}.png} \vskip\vgap
\end{figure}
\end{frame}



%--------------------------------------------------------------------------------------
\begin{frame}
\begin{figure}
\centering
\includegraphics[width=0.98\columnwidth]{{rvr_all_p140_data}.png} \vskip\vgap
\end{figure}
\end{frame}

%--------------------------------------------------------------------------------------
\begin{frame}
\begin{figure}
\centering
\includegraphics[width=\columnwidth]{{rvr_all_p140_better}.png} \vskip\vgap
\end{figure}
\end{frame}



%--------------------------------------------------------------------------------------
\begin{frame}
\frametitle{Coming Soon...}
\begin{figure}
%\frametitle{MESA profile processing}
\centering
\includegraphics[width=0.98\columnwidth]{{github_screenshot}.png} \vskip\vgap
\end{figure}
\end{frame}



\end{document} 